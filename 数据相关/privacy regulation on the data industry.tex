\documentclass{article}
\usepackage{hyperref}
\usepackage{fullpage}
\usepackage{ctex}

\begin{document}
\author{Wei HU}
\title{summary of the effect of privacy regulation on the data industry}
\date{\today}
\maketitle

本文使用了在线旅游商的数据,来探讨欧盟的数据一般条例带来的影响。GDPR使得用户自己可以选择是否追踪数据,这就使得旅游商能够观测到的消费者数量减少,
本文得到的结果是下降了12.5\%,而剩下接受了要求的客户能够被长期地观测到。本文认为,隐私强的消费者使得接受条款的消费者受到隐私外部性的影响,更容易被预测。
与本文结论一致的是,现有客户的平均营销价值更大,从而可以抵消可观测用户减少的损失。

\section*{Part1 简介}
技术发展使得公司收集数据更加简单,也因此创造了许多成功的产品。但与此同时,也出现了不好的现象。为了阻止这种不好的现象,政府应该制定政策来使得消费者对
自己的数据更有力地掌控。欧洲制定了第一部相关法律,一般数据保护条例,为其他国家的立法提供了一个蓝图。但是我们对于该法案的效力、影响缺乏实证研究,这种证据
不仅利于相关规则的制定,也有利于我们对隐私经济学的认识。

本文关心的三个问题:
\begin{itemize}
    \item 消费者在多大程度上行使GDPR赋予的同意权?
    \item GDPR如何改变企业观察到的消费者构成?
    \item GDPR隐私保护对依赖消费者数据的公司有何影响?
\end{itemize}

\subsection*{实证策略}
由于该中介与许多国家的平台建立了合作,因此它们受到GDPR的影响并不一样。而且各自平台的算法是单独训练的,因此受到GDPR影响的平台,其数据的变动并不会影响到
其他平台的算法性能。因此本文采用了DID的思路,比较欧洲国家和其他国家的结果。本文只得出了该政策的总体效用

\subsection*{得出的结论}
我们发现,GDPR导致cookie总数减少了约12.5\%,这证明消费者正在使用GDPR规定的增加的选择退出功能。然而,我们发现,剩下的消费者谁不选择退出更持久的跟踪。
\textit{可追踪性:在某个时期内,其标识符能被网站重复观测的消费者所占比例。}然后发现,GDPR实施后,可追踪性增加了8\%.

本文继续探讨了消费者可追踪性增加的两个机制:
\begin{itemize}
    \item 选择部分接受的消费者,在之前就采用了cookie block、cookie deletion和隐私浏览器等其他方法,有了GDPR后,直接选择不同意就好了。
    \item 选择性偏差问题,低频用户更不相信平台的隐私政策,而选择同意的更容易是那些平时经常使用的人。
\end{itemize}

\subsection*{新的发现}
一种新的外部性:考虑隐私的消费者对于其他部门的影响,如其他类型的消费者、企业、广告商等。
隐私意思强的消费者,其行为的改变使得其他同意条款的消费者更容易被追踪。

许多消费者拒绝条款后,第三方企业当然会遭受损失,但是由于剩下消费者是更容易被追踪的,其损失能得到缓解。
而且本文发现,虽然企业的收益是下降的,但在统计上不是显著的。说明这种缓解效应非常重要。

而这种缓解效果如何,更重要的是取决于如何利用好剩余的“良性”消费者。

\section*{Part2 制度细节和概念框架}


\section*{Part3 实证策略}

\subsection*{模型说明及其假设}
平行趋势检验:

\subsection*{潜在的问题说明}
第一点,同一家跨国公司,其在各个国家的网站设置是一致的,在欧洲国家进行调整后,在其他国家也实施同样的做法。会使得估计值比实际更小。

第二点,在实施GDPR的较短窗口期内,我们本来认为不同的国家有着相同的旅游趋势。但实际上,欧洲和美国在夏季的时候并不一样。

\end{document}
