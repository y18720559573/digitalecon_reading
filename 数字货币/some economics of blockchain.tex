\documentclass{article}
\usepackage{hyperref}
\usepackage{fullpage}
\usepackage{ctex}

\begin{document}
\title{关于区块链的简单经济学}
\author{Wei Hu}
\maketitle

\section*{Part1 区块链简介}
本文是基于经济理论来探讨了区块链技术给创新和竞争带来的影响。该技术会对两种特定成本产生影响。分别是验证成本和网络成本。
作者讨论的验证成本是关于验证状态(比如过去的交易、属性、所有权等)的成本。网络效应是指:在没有中介的情况下,引导和运作一个市场的能力。
运行的机制:验证状态不仅便宜而且还能获取经济激励(某些状态的转换),当然这个转换从网络角度出发是有利的。
这种电子市场不需要平台方就能够使得各参与方对公共基础设施进行投资,并且有如下特征:高度竞争、低进入壁垒、低隐私风险等。
当然由于去中心化的本质,还带来了新形式的低效率和治理挑战。

\section*{Part 验证成本}
中介的验证服务。并且规模越大、地理距离越远,验证服务变地越有价值。中介平台通过降低信息不对称等来给市场增加价值。

因此中介会收取费用。

这种费用会随着中介的市场势力增加越来越大。

\subsection*{验证成本的影响}

\subsubsection*{hello}
\subsubsection*{2.}

\section*{part3 网络成本}
低成本验证状态的能力,能够定义状态转移的规则。

带来的好处:
\begin{itemize}
    \item 防止权力过多地集中到创始人与早期参与者手中。
    \item 不用过多地依赖链外治理、关系合同、法律来支持运转。
    \item 较少地隐私风险,没有哪个个体能够也优先浏览网络生成的信息
    \item 
\end{itemize}

\end{document}